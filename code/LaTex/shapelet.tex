\documentclass[a4paper, 11pt]{article}

%%%%%% 导入包 %%%%%%
\usepackage{CJKutf8}
\usepackage{graphicx}
\usepackage{algorithm}
\usepackage{algorithmic}
\usepackage[unicode]{hyperref}
\usepackage{xcolor}
\usepackage{cite}
\usepackage{indentfirst}
\usepackage{booktabs}

%%%%%% 设置字号 %%%%%%
\newcommand{\chuhao}{\fontsize{42pt}{\baselineskip}\selectfont}
\newcommand{\xiaochuhao}{\fontsize{36pt}{\baselineskip}\selectfont}
\newcommand{\yihao}{\fontsize{28pt}{\baselineskip}\selectfont}
\newcommand{\erhao}{\fontsize{21pt}{\baselineskip}\selectfont}
\newcommand{\xiaoerhao}{\fontsize{18pt}{\baselineskip}\selectfont}
\newcommand{\sanhao}{\fontsize{15.75pt}{\baselineskip}\selectfont}
\newcommand{\sihao}{\fontsize{14pt}{\baselineskip}\selectfont}
\newcommand{\xiaosihao}{\fontsize{12pt}{\baselineskip}\selectfont}
\newcommand{\wuhao}{\fontsize{10.5pt}{\baselineskip}\selectfont}
\newcommand{\xiaowuhao}{\fontsize{9pt}{\baselineskip}\selectfont}
\newcommand{\liuhao}{\fontsize{7.875pt}{\baselineskip}\selectfont}
\newcommand{\qihao}{\fontsize{5.25pt}{\baselineskip}\selectfont}

%%%% 设置 section 属性 %%%%
\makeatletter
\renewcommand\section{\@startsection{section}{1}{\z@}%
{-1.5ex \@plus -.5ex \@minus -.2ex}%
{.5ex \@plus .1ex}%
{\normalfont\sihao\CJKfamily{hei}}}
\makeatother

%%%% 设置 subsection 属性 %%%%
\makeatletter
\renewcommand\subsection{\@startsection{subsection}{1}{\z@}%
{-1.25ex \@plus -.5ex \@minus -.2ex}%
{.4ex \@plus .1ex}%
{\normalfont\xiaosihao\CJKfamily{hei}}}
\makeatother

%%%% 设置 subsubsection 属性 %%%%
\makeatletter
\renewcommand\subsubsection{\@startsection{subsubsection}{1}{\z@}%
{-1ex \@plus -.5ex \@minus -.2ex}%
{.3ex \@plus .1ex}%
{\normalfont\xiaosihao\CJKfamily{hei}}}
\makeatother

%%%% 段落首行缩进两个字 %%%%
\makeatletter
\let\@afterindentfalse\@afterindenttrue
\@afterindenttrue
\makeatother
\setlength{\parindent}{2em}  %中文缩进两个汉字位


%%%% 下面的命令重定义页面边距,使其符合中文刊物习惯 %%%%
\addtolength{\topmargin}{-54pt}
\setlength{\oddsidemargin}{0.63cm}  % 3.17cm - 1 inch
\setlength{\evensidemargin}{\oddsidemargin}
\setlength{\textwidth}{14.66cm}
\setlength{\textheight}{24.00cm}    % 24.62

%%%% 下面的命令设置行间距与段落间距 %%%%
\linespread{1.4}
% \setlength{\parskip}{1ex}
\setlength{\parskip}{0.5\baselineskip}

%%%% 正文开始 %%%%
\begin{document}
\begin{CJK}{UTF8}{gbsn}

%%%% 定理类环境的定义 %%%%
\newtheorem{example}{例}             % 整体编号
%\newtheorem{algorithm}{算法}
\newtheorem{theorem}{定理}[section]  % 按 section 编号
\newtheorem{definition}{定义}
\newtheorem{axiom}{公理}
\newtheorem{property}{性质}
\newtheorem{proposition}{命题}
\newtheorem{lemma}{引理}
\newtheorem{corollary}{推论}
\newtheorem{remark}{注解}
\newtheorem{condition}{条件}
\newtheorem{conclusion}{结论}
\newtheorem{assumption}{假设}

%%%% 重定义 %%%%
\renewcommand{\contentsname}{目录}  % 将Contents改为目录
\renewcommand{\abstractname}{摘要}  % 将Abstract改为摘要
\renewcommand{\refname}{参考文献}   % 将References改为参考文献
\renewcommand{\indexname}{索引}
\renewcommand{\figurename}{图}
\renewcommand{\tablename}{表}
\renewcommand{\appendixname}{附录}
%\renewcommand{\algorithm}{算法}


%%%% 定义标题格式,包括title,author,affiliation,email等 %%%%
\title{shapelet\\some thoughts}
\author{kaikaifeng\footnote{E-mail: kaikaifengqh@163.com,Student ID: xxxxxxxx}\\[2ex]
\xiaosihao BUAA\\[2ex]
}
\date{2017-07-29}


%%%% 以下部分是正文 %%%%
\maketitle

%\tableofcontents
\newpage

\section{shapelet——时间序列分类}
\subsection{简单综述}
时间序列分类问题(Time-series classification or TSC)的困难之处在于如何度量序列之间的相似性。传统的分类问题中,属性的顺序通常是不重要的;然而对于时间序列,数据的排序对于找到好的分类特征至关重要。对TSC的研究(有一部分)集中在寻找最近邻(Nearest Neighbor or NN)分类器使用的距离度量。实际上,在小数据集上,NN(虽然看上去naive)效果“很好”:"There is a plethora of classification algorithms that can be applied to time series; however, all of the current empirical evidence suggests that simple nearest neighbor classification is very difficult to beat" \cite{Batista}。\par

最近邻分类器适用于那种通常的时域曲线\ref{fig:fig1}。曲线的(潜在的)基本形状的变动被认为是观察时引入的噪声导致的。由噪声导致的相位变化较小。\par

可以不太准确地说,最近邻以及其他关注完整曲线的方法是“全局方法”。\par

\begin{figure}[htbp]
    \small
    \centering
    \includegraphics[width=15cm]{2.JPG}
    \label{fig:fig1}
    \caption{全局匹配\cite{Hills}}
\end{figure}

\subsection{shapelet——关注局部}
shapelet方法不直接关注时间序列的全局特征,而试图寻找序列间局部的相似性。一个shapelet的简介定义是:"A shapelet is a time-series subsequence that can be used as a primitive for TSC based on \textbf{local}, phase-independent similarity in shape"\cite{Hills}\par

shapelet的基本思想也并不复杂,如图\ref{fig:fig2}所示,一个shapelet应能够和训练集中的一部分时间序列的某些连续子序列匹配地很好,然而在剩余的时间序列中却不能找到匹配良好的连续子序列。\par

\begin{figure}[htbp]
    \small
    \centering
    \includegraphics[width=15cm]{1.JPG}\\
    \label{fig:fig2}
    \caption{局部匹配\cite{Lexiang}}
\end{figure}

下表\ref{tab:symbolTable}说明了将要使用的符号。\par

\begin{table}[htbp]
    \centering
    \caption{符号表}
        \label{tab:symbolTable}
    \begin{tabular}{cc}
        \toprule
        符号 & 说明 \\
        \midrule
        $T, R$  &   时间序列\\
        $S$     &   连续子序列\\
        $m, |T|$    &   时间序列长度\\
        $l, |S|$    &   连续子序列长度\\
        $\mathbf{D}$    &   时间序列数据集\\
        $A, B$  &   类别标签\\
        $S_{{k}}$   &   $S$中的第$k$个数据点\\
        $\mathbf{S}_{T}^{|S|}$  &   $T$的长度为$|S|$的连续子序列集合\\
        \bottomrule
    \end{tabular}
\end{table}

\definition :时间序列和连续子序列的距离。以$Dist(T, R)$表示两个等长的时间序列$T, R$的距离,时间序列和连续子序列的距离$SubsequenceDist(T, S)$ 定义为:
\[SubsequenceDist(T, S) = min(Dist(S, S^{'})), S^{'} \in \mathbf{S}_{T}^{|S|}\]\par

由此可见,单纯的shapelet关注的是完全与相位无关的“形状”特征,这还不是我们想要的。\par

\subsection{shapelet——分类器}
shapelet的提出最早是用于分类的。其基本算法如下\ref{alg:algorithm}:\par

\begin{algorithm}[htbp]
    \centering
    \caption{FindingShapeletBF(dataset \textbf{D}, \textit{MAXLEN, MINLEN})}
        \label{alg:algorithm}
    \begin{algorithmic}[1]
        \STATE candidates $\leftarrow$ GenerateCandidates(\textbf{D}, MAXLEN,MINLEN)\\
        \STATE bsf\_gain $\leftarrow 0$ \\
        \STATE \textbf{For each} S \textbf{in} candidates\\
        \STATE \qquad gain $\leftarrow$ CheckCandidate(\textbf{D}, S)\\
        \STATE \qquad \textbf{If} gain > bsf\_gain\\
        \STATE \qquad \qquad bsf\_gain $\leftarrow$ gain\\
        \STATE \qquad \qquad bsf\_shapelet $\leftarrow$ S\\
        \STATE \qquad \textbf{Endif}\\
        \STATE \textbf{Endfor}\\
        \STATE \textbf{Return} bsf\_shaplet\\
    \end{algorithmic}
\end{algorithm}

算法首先生成一些候选序列,然后使用信息增益(\textit{Information Gain})衡量候选序列的分类效果,并找到信息增益最大的候选序列。对集合的划分利用时间序列和连续子序列的距离$SubsequenceDist(T, S)$,如图所示\ref{fig:fig3},CheckCandidiate()函数中,首先计算\textbf{D}中所有序列与候选shapelet之间的距离,然后找到最佳的划分点,并得到对应的信息增益。\par

\begin{figure}[htbp]
    \small
    \centering
    \includegraphics[width=15cm]{3.JPG}
    \label{fig:fig3}
    \caption{划分点}
\end{figure}

划分点的选取需要遍历得到的距离序列中相邻点对的中点。\par

由基本算法\ref{alg:algorithm}可以构造决策树,用于时间序列的分类。这是最初提出的shapelet用法。\par

%参考文献的格式可能有问题,这不是用于投稿的材料,只用于记录、交流
\newpage
\begin{thebibliography}{}
\bibitem[Batista et al.(2011)]{Batista} Batista G, Wang X, Keogh E (2011) A complexity-invariant distance measure for time series. Proceedings of the eleventh SIAM conference on data mining (SDM)

\bibitem[Lexiang Ye \& Eamonn Keogh(2009)]{Lexiang} Lexiang Ye, Eamonn Keogh (2009) Time Series Shapelets: A New Primitive for Data Mining. ACM Knowledge Discovery and Data Mining (KDD)

\bibitem[Hills et al.(2013)]{Hills} Jon Hills · Jason Lines · Edgaras Baranauskas James Mapp · Anthony Bagnall (2013) Classification of time series by shapelet transformation: Data Mining and Knowledge
Discovery (DMKD)
\end{thebibliography}

\end{CJK}
\end{document}
