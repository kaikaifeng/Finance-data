\documentclass[a4paper, 11pt]{article}

%%%%%% 导入包 %%%%%%
\usepackage{CJKutf8}
\usepackage{graphicx}
\usepackage[unicode]{hyperref}
\usepackage{xcolor}
\usepackage{cite}
\usepackage{indentfirst}
\usepackage{booktabs}
\usepackage{amsfonts}
\usepackage{amsmath}

%%%%%% 设置字号 %%%%%%
\newcommand{\chuhao}{\fontsize{42pt}{\baselineskip}\selectfont}
\newcommand{\xiaochuhao}{\fontsize{36pt}{\baselineskip}\selectfont}
\newcommand{\yihao}{\fontsize{28pt}{\baselineskip}\selectfont}
\newcommand{\erhao}{\fontsize{21pt}{\baselineskip}\selectfont}
\newcommand{\xiaoerhao}{\fontsize{18pt}{\baselineskip}\selectfont}
\newcommand{\sanhao}{\fontsize{15.75pt}{\baselineskip}\selectfont}
\newcommand{\sihao}{\fontsize{14pt}{\baselineskip}\selectfont}
\newcommand{\xiaosihao}{\fontsize{12pt}{\baselineskip}\selectfont}
\newcommand{\wuhao}{\fontsize{10.5pt}{\baselineskip}\selectfont}
\newcommand{\xiaowuhao}{\fontsize{9pt}{\baselineskip}\selectfont}
\newcommand{\liuhao}{\fontsize{7.875pt}{\baselineskip}\selectfont}
\newcommand{\qihao}{\fontsize{5.25pt}{\baselineskip}\selectfont}

%%%% 设置 section 属性 %%%%
\makeatletter
\renewcommand\section{\@startsection{section}{1}{\z@}%
{-1.5ex \@plus -.5ex \@minus -.2ex}%
{.5ex \@plus .1ex}%
{\normalfont\sihao\CJKfamily{hei}}}
\makeatother

%%%% 设置 subsection 属性 %%%%
\makeatletter
\renewcommand\subsection{\@startsection{subsection}{1}{\z@}%
{-1.25ex \@plus -.5ex \@minus -.2ex}%
{.4ex \@plus .1ex}%
{\normalfont\xiaosihao\CJKfamily{hei}}}
\makeatother

%%%% 设置 subsubsection 属性 %%%%
\makeatletter
\renewcommand\subsubsection{\@startsection{subsubsection}{1}{\z@}%
{-1ex \@plus -.5ex \@minus -.2ex}%
{.3ex \@plus .1ex}%
{\normalfont\xiaosihao\CJKfamily{hei}}}
\makeatother

%%%% 段落首行缩进两个字 %%%%
\makeatletter
\let\@afterindentfalse\@afterindenttrue
\@afterindenttrue
\makeatother
\setlength{\parindent}{2em}  %中文缩进两个汉字位


%%%% 下面的命令重定义页面边距,使其符合中文刊物习惯 %%%%
\addtolength{\topmargin}{-54pt}
\setlength{\oddsidemargin}{0.63cm}  % 3.17cm - 1 inch
\setlength{\evensidemargin}{\oddsidemargin}
\setlength{\textwidth}{14.66cm}
\setlength{\textheight}{24.00cm}    % 24.62

%%%% 下面的命令设置行间距与段落间距 %%%%
\linespread{1.4}
% \setlength{\parskip}{1ex}
\setlength{\parskip}{0.5\baselineskip}

%%%% 正文开始 %%%%
\begin{document}
\begin{CJK}{UTF8}{gbsn}

%%%% 定理类环境的定义 %%%%
\newtheorem{example}{例}             % 整体编号
\newtheorem{algorithm}{算法}
\newtheorem{theorem}{定理}[section]  % 按 section 编号
\newtheorem{definition}{定义}
\newtheorem{axiom}{公理}
\newtheorem{property}{性质}
\newtheorem{proposition}{命题}
\newtheorem{lemma}{引理}
\newtheorem{corollary}{推论}
\newtheorem{remark}{注解}
\newtheorem{condition}{条件}
\newtheorem{conclusion}{结论}
\newtheorem{assumption}{假设}

%%%% 重定义 %%%%
\renewcommand{\contentsname}{目录}  % 将Contents改为目录
\renewcommand{\abstractname}{摘要}  % 将Abstract改为摘要
\renewcommand{\refname}{参考文献}   % 将References改为参考文献
\renewcommand{\indexname}{索引}
\renewcommand{\figurename}{图}
\renewcommand{\tablename}{表}
\renewcommand{\appendixname}{附录}
%\renewcommand{\algorithm}{算法}


%%%% 定义标题格式,包括title,author,affiliation,email等 %%%%
\title{GAN\\}
\author{kaikaifeng\footnote{E-mail: kaikaifengqh@163.com,Student ID: xxxxxxxx}\\[2ex]
\xiaosihao BUAA\\[2ex]
}
\date{2017-07-29}


%%%% 以下部分是正文 %%%%
\maketitle

%\tableofcontents
\newpage

\noindent 符号说明:\\*
$x$是分类器输入空间的随机变量,“样本所在空间”中的随机变量
$z$是用作生成器输入的随机变量

下面是\textit{Generative Adversarial Nets}中结论的一些推导和说明\\*
$\mathbf{V(G, D)}$\\*是什么?
模型的目标是通过$D$和$G$的训练最小最大化$V(G, D)$,文中写作:
\begin{equation}
\min\limits_{G}\max\limits_{D}V(D, G) = \mathbb{E}_{x\sim p_{data}(X)}[\log D(x)] + \mathbb{E}_{z\sim p_{z}(z)}[\log(1 - D(G(z)))]
\end{equation}
其中$\mathbb{E}$是数学期望,所以将上式依照定义展开,得到:
\begin{equation}
V(G, D) = \int_{x}p_{data}(x)log(D(x))dx +\int_{z}p_{z}(z)log(1-D(G(z)))dz
\end{equation}
对于确定的$G$,什么样的$D$是最好的?\\*
在$(2)$中,$p_{data}$是数据集的概率密度函数,$p_{z}$是随机噪声$z$的概率密度函数,而模型的目的是得到一个生成器$G$,或者说一个函数$G$,使得随机变量$z$的函数$G(z)$,这也是一个随机变量(不去考虑$G(z)$是否总是一个随机变量), 记$p_{g}$是$G(z)$的概率密度函数。希望能够使得$p_{data}$和$p_{g}$尽可能相似\\*
根据上述的符号含义,观察$(2)$加号的右侧,在$z$空间中$log(1-D(G(z)))$的数学期望就是$x$空间中的$log(1-D(x))$,相应的,概率密度函数从$p_{z}$变为$p_{g}$,即有:
\begin{equation}
V(G,D) = \int_{x}p_{data}(x)log(D(x)) + p_{g}(x)log(1-D(x))dx
\end{equation}
我们考虑如何让$V(G,D)$最大,注意到$D$是一个函数,所以最好的情况下,积分的内部总是取到最大值,按照文中的符号,有:
\[p_{data}(x)log(D_{G}^{*}(x)) + p_{g}(x)log(1-D_{G}^{*}(x))\]
总是取得最大值,而$a\log(y)+b\log(1-y)$形式的函数在$\frac{a}{a+b}$处取得最大值,所以有,给定$G$,最好的$D$是
\begin{equation}
D_{G}^{*} = \frac{p_{data}(x)}{p_{data}(x)+p_g(x)}
\end{equation}
\textbf{如果生成器足够好,}$p_{data}$\textbf{和}$p_{g}$\textbf{几乎相同},此时,$V(G,D)=-log(4)$,下面来说明这实际上就是$V(G,D)$的最小最大值,而且,当且仅当$p_{data}=p_{g}$是取得(实际上,可能应该写作几乎处处成立时。。。)
\[
\begin{split}
V(G,D_{G}^{*})&=\int_{x}p_{data}(x)log(D_{G}^{*}(x)) + p_{g}(x)log(1-D_{G}^{*}(x))dx\\
&=\int_{x}p_{data}(x)log(\frac{p_{data}(x)}{p_{data}(x)+p_{g}(x)})+p_{g}(x)log(\frac{p_{g}(x)}{p_{data}(x)+p_{g}(x)})dx\\
&=\int_{x}p_{data}(x)(log(\frac{2p_{data}(x)}{p_{data}(x)+p_{g}(x)}-log(2)))+p_{g}(x)(log(\frac{2p_{g}(x)}{p_{data}(x)+p_{g}(x)})-log(2)))dx\\
&=-log(4)+\int_{x}p_{data}(x)log(\frac{2p_{data}(x)}{p_{data}(x)+p_{g}(x)})+p_{g}(x)log(\frac{2p_{g}(x)}{p_{data}(x)+p_{g}(x)})dx
\end{split}
\]
我们观察一下积分符号内部,注意到$-log(x)$满足
\[-log(\lambda x_{1}+(1-\lambda)x_{2})\leq-\lambda\log(x_{1})-(1-\lambda)\log(x_{2})\]
我们有:
\[
\begin{split}
p_{data}(x)log(\frac{2p_{data}(x)}{p_{data}(x)+p_{g}(x)})+p_{g}(x)log(\frac{2p_{g}(x)}{p_{data}(x)+p_{g}(x)})=\\
(p_{data}+p_{g})\left[\frac{p_{data}}{p_{data}+p_{g}}\cdot(-log(\frac{p_{data}+p_{g}}{2p_{data}}))+\frac{p_{g}}{p_{data}+p_{g}}\cdot(-log(\frac{p_{data}+p_{g}}{2p_{g}}))
\right]\geq\\
(p_{data}+p_{g})\cdot-log\left(\frac{p_{data}}{p_{data}+p_{g}}\cdot\frac{p_{data}+p_{g}}{2p_{data}}+\frac{p_{g}}{p_{data}+p_{g}}\cdot\frac{p_{data}+p_{g}}{2p_{g}}\right)=\\
0
\end{split}
\]
又因为$p_{data},p_{g}$总是非负,所以
\[\int_{x}p_{data}(x)log(\frac{2p_{data}(x)}{p_{data}(x)+p_{g}(x)})+p_{g}(x)log(\frac{2p_{g}(x)}{p_{data}(x)+p_{g}(x)})dx \geq0\]
当且仅当$p_{data}=p_{g}$时等号成立,得证

\end{CJK}
\end{document}
