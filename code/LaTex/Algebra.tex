\documentclass[a4paper, 11pt]{article}

%%%%%% 导入包 %%%%%%
\usepackage{CJKutf8}
\usepackage{graphicx}
\usepackage[unicode]{hyperref}
\usepackage{xcolor}
\usepackage{cite}
\usepackage{indentfirst}

%%%%%% 设置字号 %%%%%%
\newcommand{\chuhao}{\fontsize{42pt}{\baselineskip}\selectfont}
\newcommand{\xiaochuhao}{\fontsize{36pt}{\baselineskip}\selectfont}
\newcommand{\yihao}{\fontsize{28pt}{\baselineskip}\selectfont}
\newcommand{\erhao}{\fontsize{21pt}{\baselineskip}\selectfont}
\newcommand{\xiaoerhao}{\fontsize{18pt}{\baselineskip}\selectfont}
\newcommand{\sanhao}{\fontsize{15.75pt}{\baselineskip}\selectfont}
\newcommand{\sihao}{\fontsize{14pt}{\baselineskip}\selectfont}
\newcommand{\xiaosihao}{\fontsize{12pt}{\baselineskip}\selectfont}
\newcommand{\wuhao}{\fontsize{10.5pt}{\baselineskip}\selectfont}
\newcommand{\xiaowuhao}{\fontsize{9pt}{\baselineskip}\selectfont}
\newcommand{\liuhao}{\fontsize{7.875pt}{\baselineskip}\selectfont}
\newcommand{\qihao}{\fontsize{5.25pt}{\baselineskip}\selectfont}

%%%% 设置 section 属性 %%%%
\makeatletter
\renewcommand\section{\@startsection{section}{1}{\z@}%
{-1.5ex \@plus -.5ex \@minus -.2ex}%
{.5ex \@plus .1ex}%
{\normalfont\sihao\CJKfamily{hei}}}
\makeatother

%%%% 设置 subsection 属性 %%%%
\makeatletter
\renewcommand\subsection{\@startsection{subsection}{1}{\z@}%
{-1.25ex \@plus -.5ex \@minus -.2ex}%
{.4ex \@plus .1ex}%
{\normalfont\xiaosihao\CJKfamily{hei}}}
\makeatother

%%%% 设置 subsubsection 属性 %%%%
\makeatletter
\renewcommand\subsubsection{\@startsection{subsubsection}{1}{\z@}%
{-1ex \@plus -.5ex \@minus -.2ex}%
{.3ex \@plus .1ex}%
{\normalfont\xiaosihao\CJKfamily{hei}}}
\makeatother

%%%% 段落首行缩进两个字 %%%%
\makeatletter
\let\@afterindentfalse\@afterindenttrue
\@afterindenttrue
\makeatother
\setlength{\parindent}{2em}  %中文缩进两个汉字位


%%%% 下面的命令重定义页面边距,使其符合中文刊物习惯 %%%%
\addtolength{\topmargin}{-54pt}
\setlength{\oddsidemargin}{0.63cm}  % 3.17cm - 1 inch
\setlength{\evensidemargin}{\oddsidemargin}
\setlength{\textwidth}{14.66cm}
\setlength{\textheight}{24.00cm}    % 24.62

%%%% 下面的命令设置行间距与段落间距 %%%%
\linespread{1.4}
% \setlength{\parskip}{1ex}
\setlength{\parskip}{0.5\baselineskip}

%%%% 正文开始 %%%%
\begin{document}
\begin{CJK}{UTF8}{gbsn}

%%%% 定理类环境的定义 %%%%
\newtheorem{example}{例}             % 整体编号
\newtheorem{algorithm}{算法}
\newtheorem{theorem}{定理}[section]  % 按 section 编号
\newtheorem{definition}{定义}
\newtheorem{axiom}{公理}
\newtheorem{property}{性质}
\newtheorem{proposition}{命题}
\newtheorem{lemma}{引理}
\newtheorem{corollary}{推论}
\newtheorem{remark}{注解}
\newtheorem{condition}{条件}
\newtheorem{conclusion}{结论}
\newtheorem{assumption}{假设}

%%%% 重定义 %%%%
\renewcommand{\contentsname}{目录}  % 将Contents改为目录
\renewcommand{\abstractname}{摘要}  % 将Abstract改为摘要
\renewcommand{\refname}{参考文献}   % 将References改为参考文献
\renewcommand{\indexname}{索引}
\renewcommand{\figurename}{图}
\renewcommand{\tablename}{表}
\renewcommand{\appendixname}{附录}
\renewcommand{\algorithm}{算法}


%%%% 定义标题格式,包括title,author,affiliation,email等 %%%%
\title{\textbf{Tensor}\\note token from \textit{Algebra} \textbf{Serge lang}}
\author{your name\footnote{E-mail: your@email.com,Student ID: your ID}\\[2ex]
\xiaosihao BUAA\\[2ex]
}
\date{Date}


%%%% 以下部分是正文 %%%%
\maketitle

%\tableofcontents
\newpage

This part introduces the basic notions of algebra, and the main difficulty for the beginner is to absorb a reasonable vocabulary in a short time.
None of the concepts is difficult, but there is an accumulation of new concepts which may sometimes seem heavy.\par
To understand the next parts of the book, the reader needs to know essentially only the basic definitions of this first part. Of course, a theorem may be used later for
some specific and isolated applications, but on the whole, we have avoided making long logic chains of interdependence.  ---\textbf{Serge Lang}

Let $S$ be a set. A mapping
\[S \times S \longrightarrow S\]
is called a \textbf{law of composition}. If $x, y$ are elements of $S$, the image of the pair $(x, y)$ under this mapping is called their \textbf{product} under the 
law of composition, and will be denoted by $xy$ or $x \cdot y$ or $x + y$ (only when the relation $x + y = y + x$ holds).\par

Let $S$ be a set with a law of composition and $x, y, z$ are elements of $S$. If $(xy)z = x(yz)$ for all $x, y, z$ in $S$ then we say that the law of composition is
\textbf{associative}.\par

An element $\emph{e}$ of $S$ such that $\emph{e}x = x = x\emph{e}$ for all $x \in S$ is called a \textbf{unit element}
(when the law of composition is written additively, the unit element is denoted by $0$, and is called a \textbf{zero element}).
A unit element is unique, for we have
\[\emph{e} = \emph{e}\emph{e}^{'} = \emph{e}^{'}\] 
In most cases, the unit element is written $1$.\par

A \textbf{monoid} is a set $G$, with a law of composition which is associative, and having a unit element.\par

The product of elements of $G$ are defined inductively:
\[\prod_{\nu = 1}^{n}x_{\nu} = (x_{1} \dots x_{n - 1})x_{n}\]
We then have the following rule:
\[\prod_{\mu = 1}^{m}x_{\mu} \cdot \prod_{\nu = 1}^{n}x_{m + \nu} = \prod_{\nu = 1}^{m + n}x_{\nu}\]
We define
\[\prod_{\nu = 0}^{0}x_{\nu} = \emph{e}\]
As a matter of convention, we agree also that the empty product is equal to the unit element\par

For a law of composition
\[f: S \times S \longrightarrow T\]
\textbf{Commutativity} means $f(x, y) = f(y, x)$ or $xy = yx$\par

If the law of composition of $G$ is commutative, we also say that $G$ is \textbf{commutative(or abelian)}\par

let $G$ be a commutative monoid, and $x_{1}, \dots, x_{n}$ elements of $G$. Let $\psi$ be a bijection of the set of integers $(1, \dots, n)$
onto itself. Then
\[\prod_{nu = 1}^{n}x_{\psi(\nu)} = \prod_{\nu = 1}^{n}x_{\nu}\]

When $G$ is written additively, then instead of a product sign, we write the sum sign $\Sigma$.\par

Let $x$ be an element of a monoid $G$. For every integer $n \geq 0$ we define $x^{n}$ to be 
\[\prod_{1}^{n}x\]
We obviously have $x^{n + m} = x^{n}x^{m}$ and $(x^{n})^{m} = x^{nm}$. From our preceding rules of associativity and commutativity, if $x, y$ are elements 
of $G$ such that $xy = yx$, then $(xy)^{n} = x^{n}y^{n}$\par

A \textbf{group} is a monoid, such that for every element $x \in G$ there exists an element $y \in G$ such that $xy = yx = \emph{e}$. Such an element $y$ is called an
\textbf{inverse} for $x$. Such an inverse is unique, because
\[y^{'} = y^{'}\emph{e} = y^{'}(xy) = (y^{'}x)y = \emph{e}y = y\]
We denote this inverse by $x^{-1}$ or $-x$ and for any positive integer $n$, we let $x^{-n} = (x^{-1})^{n}$\par

Let $G$ be a set with associative law of composition, let \emph{e} be a left unit for that law, and assume that every element has a left inverse. Then \emph{e} is a unit, and each left inverse is also an inverse.
let $ba = e$. Then
\[bab = \emph{e}b = b\]
Multiplying on the left by a left inverse for $b$ yields
\[ab = \emph{e}\]
Furthermore,
\[a\emph{e} = aba = \emph{e}a = a\]
whence \emph{e} is a right unit\par

Let $G, G^{'}$ be monoids. A \textbf{monoid-homomorphism} of $G$ into $G^{'}$ is a mapping $f: G \longrightarrow G^{'}$ such that $f(xy) = f(x)f(y)$ for all $x, y \in G$. If $G, G^{'}$ are groups, a \textbf{group-homomorphism} of $G$ into $G^{'}$ is simply a monoid-homomorphism.\par

Let $f: G \longrightarrow G^{'}$ be a group-homomorphism. Then
\[f(x^{-1} = f(x)^{-1})\]
Proof:
\[f(\emph{e}) = f(\emph{ee}) = f(\emph{e})f(\emph{e})\]
so we get $\emph{e}^{'} = f(\emph(e))$
\[\emph{e}^{'} = f(\emph{e}) = f(xx^{-1}) = f(x)f(x^{-1})\]
then
\[f(x^{-1}) = f(x)^{-1}\]
\end{CJK}
\end{document}
